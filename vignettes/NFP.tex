%\VignetteIndexEntry{mmnet}
%\VignetteKeywords{mmnet}
%\VignettePackage{mmnet}
%\VignetteEngine{knitr::knitr}

\documentclass[a4paper]{article}
\usepackage{times}
\usepackage{natbib}
\usepackage{hyperref}
\usepackage{amsmath}
%\usepackage{Sweave}
\usepackage[utf8]{inputenc}
%\usepackage[pdftex]{graphicx}
\usepackage{url}


\title{NFP:Network fingerprint - a knowledge-based characterization of biomedical networks}
\author{Yang Cao, Fei Li, Xiaochen Bo}

\begin{document}
\input{NFP-concordance}

\begin{Schunk}
\begin{Sinput}
> library(knitr)
> options(width=64,digits=2)
> opts_chunk$set(size="small")
> opts_chunk$set(tidy=TRUE,tidy.opts=list(width.cutoff=50,keep.blank.line=TRUE))
> opts_knit$set(eval.after='fig.cap')
> # for a package vignette, you do want to echo.
> # opts_chunk$set(echo=FALSE,warning=FALSE,message=FALSE)
> opts_chunk$set(warning=FALSE,message=FALSE)
> opts_chunk$set(cache=TRUE,cache.path="cache/NFP")
\end{Sinput}
\end{Schunk}

\bibliographystyle{plainnat}

%\SweaveOpts{highlight=TRUE, tidy=TRUE, keep.space=TRUE, keep.blank.space=FALSE, keep.comment=TRUE}
\maketitle
\tableofcontents

\section{Introduction}

This manual is a brief introduction to structure, functions and usage of
\emph{NFP} package. The \emph{NFP} package provides a set of functions to
support Network fingerprint algorithm: a biomedical network is characterized as a spectrum-like vector called “network fingerprint”.



\subsection{Installation}


\emph{NFP} requires these packages: \emph{magrittr}, \emph{igraph}, \emph{plyr},
\emph{ggplot2}, \emph{apcluster}, \emph{dplyr}, \emph{stringr}, \emph{plyr} and
\emph{KEGGgraph}. Since \emph{NFP} is on CRAN or Bioconductor, users can install
the latest development version from github. To install packages from GitHub, you
first need install the devtools package on your system with
`install.packages("devtools")`. Note that devtools sometimes needs some extra
non-R software on your system -- more specifically, an Rtools download for
Windows or Xcode for OS X.

